\documentclass[review]{elsarticle}

\usepackage{lineno,hyperref}
\modulolinenumbers[5]

\journal{Nuclear Instruments and Methods B}

%%%%%%%%%%%%%%%%%%%%%%%
%% Elsevier bibliography styles
%%%%%%%%%%%%%%%%%%%%%%%
%% To change the style, put a % in front of the second line of the current style and
%% remove the % from the second line of the style you would like to use.
%%%%%%%%%%%%%%%%%%%%%%%

%% Numbered
%\bibliographystyle{model1-num-names}

%% Numbered without titles
%\bibliographystyle{model1a-num-names}

%% Harvard
%\bibliographystyle{model2-names.bst}\biboptions{authoryear}

%% Vancouver numbered
%\usepackage{numcompress}\bibliographystyle{model3-num-names}

%% Vancouver name/year
%\usepackage{numcompress}\bibliographystyle{model4-names}\biboptions{authoryear}

%% APA style
%\bibliographystyle{model5-names}\biboptions{authoryear}

%% AMA style
%\usepackage{numcompress}\bibliographystyle{model6-num-names}

%% `Elsevier LaTeX' style
\bibliographystyle{elsarticle-num}
%%%%%%%%%%%%%%%%%%%%%%%

\begin{document}

\begin{frontmatter}

\title{Liquid scintillator tiles for high radiation environments }


%% or include affiliations in footnotes:
\author[umd]{Alberto Belloni\corref{mycorrespondingauthor}}
\cortext[mycorrespondingauthor]{Corresponding author}
\ead{abelloni@umd.edu}
\author[umd]{Mahnegar Amouzegar}
\author[umd]{Jeff Calderon}
\author[umd]{Sarah C. Eno}
\author[baylor]{Kenichi Hatakeyama}
\author[fnal]{Kevin Pedro}
\author[umd]{Geng Yuan Jeng}
\author[umd]{Joshua Samuel}
\author[elmer]{Elmer Sharp}
\author[umd]{Young Ho Shin}
\author[umd]{Zishuo Yang}
\author[umd]{Yao Yao}
\author[korea]{Sung Woo Youn}




\address[umd]{Dept. Physics, U. Maryland, College Park MD 30742 USA}
\address[eljen]{Eljen Technology, 1300 W. Broadway, Sweetwater, Tx 79556 USA}
\address[korea]{Institute for Basic Science, Center for Axion and Precision Physics Research, IBS Center for Axion and Precision Physics Research
Room 4315, Department of Physics, Natural Science Building (E6-2), KAIST,
291 Daehak-ro, Yuseong-gu, Daejeon 305-701, South Korea}
\address[elmer]{Elmer Sharp Engineering, 7007 Leesville Blvd. Springfield, VA 22151}
\address[fnal]{Fermi National Accelerator Laboratory, Batavia, IL, USA}
\address[baylor]{Baylor University, Waco, Texas, USA}

\begin{abstract}
Future experiments in high energy and nuclear physics may require large, inexpensive calorimetery that can operate to doses of 50 Mrad or more.
We present the results of a study of a scintillator tile based on EJ-309 liquid scintillator using cosmic rays, test beam, and $\rm{^{60}Co}$ irradiations. 
\end{abstract}

\begin{keyword}
organic scintillator\sep liquid scintillator\sep \sep radiation hardness \sep calorimetry
\end{keyword}

\end{frontmatter}

\linenumbers

\section{Introduction}
Sampling calorimeters using plastic scintillator tiles with wave length shifting fibers, such as the CDF plug calorimeter \cite{Albrow20022524}, are popular due to their excellent performance at a reasonable cost.  Plastic scintillator is available commercially from companies like Kuraray, St. Gobain, and Eljen.  When irradiated, however, the performance of plastic scintillator deteriorates; light self-absorption (yellowing) increases and light output decreases.  The resulting damage has been studied for most common plastics\cite{34504}, \cite{Wick1991472}, \cite{289295},
\cite{173180},\cite{467829},\cite{Wulkop1995141},\cite{173178},\cite{vasken}.  Generally, the light output decreases with exponentially with dose, with an decay constant on order of a few Mrad.  Future high energy and nuclear experiments, however, may have to operate in environments that will deliver doses of tens of Mrad.  In this paper, we present the design and optimization of a liquid scintillator tile, based on EJ-309 liquid scintillator, that can operate in thie kind of environment.


\section{Tile design}

\section{Test beam results}

\section{Light yield dependence on tile parameters and comparison with simulation}

\section{Radiation hardness tests}


\section{Conclusions}

\section{Acknowledgements}
The authors would like to thank Randy Ruchti of Notre Dame for providing the capillaries and Yasar Onel's group at the University of Iowa for help with the test beam.
This work was supported in part by U.S. Department of Energy Grant YYYYY.

\section*{References}

\bibliography{aliquidtile}

\end{document}